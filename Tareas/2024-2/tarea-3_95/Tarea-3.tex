\documentclass[spanish, draft]{article}
\usepackage{amsmath, amsfonts, amsthm}
\begin{document}

\title{
    Tarea 3 \\
    Informatica Teorica
}
\author{
    Matias peñaloza
    202373037-8
}
\date{
    2024-2
}
\maketitle

\begin{center}
    \begin{tabular}{|l|r|}
      \hline
      \multicolumn{1}{|c|}{\textbf{Concepto}} &
        \multicolumn{1}{c|}{\textbf{Tiempo [min]}} \\
      \hline
      Revisión & 10\\
      \hline
      Desarrollo    & 30\\
      \hline
      Informe	      & 30\\
      \hline
    \end{tabular}
  \end{center}

\section{Enunciado}
    Dé gramáticas de contexto libre para los lenguajes siguientes.
    Explique brevemente sus diseños.
    \begin{enumerate}
    \item % 20242t3p1
    \(L_1 = \{a^m b^n c^{m + n} \colon m, n \in \mathbb{N}\}\)
    \item % 20242t3p2
    \(L_2 = \{a^m b^m c^n d^n \colon m, n \in \mathbb{N}\}\)
    \item % 20242t3p3
    \(L_3 = \{\omega \in \{a, b\}^*
                \colon \lvert \omega \rvert
                        \text{\ es impar y el símbolo central es \(a\)}\}\)
    \end{enumerate}

\section{Desarollo}
    Para evitar confusion es clave aclarar que en esta entrega se considera:
    \begin{enumerate}
        \item $0 \notin \mathbb{N}$
        \item $0 \neq 2n \pm 1, \forall n \in \mathbb{Z} \Rightarrow 0$ no es impar
    \end{enumerate}
    una vez aclarado esto se presentan a continuacion las gramaticas pedidas en el enunciado.

    \subsection{Gramatica $G_1$ para $L_1$}
        Podemos notar que para formar palabras del tipo $a^mb^nc^{m+n} = a^mb^nc^nc^m$
        unicamente necesitamos generar la misma cantidad de $a's$ que de $c's$ y la misma cantidad de $b's$ que de $c's$,
        entonces para las $a's$ tendriamos algo asi:
        $$S \longrightarrow aSc$$
        Y para las $b's$ algo asi:
        $$T \longrightarrow bTc$$
        Ahora la seccion $b^nc^n$ esta entre la seccion $a^m$ y la seccion $c^m$ por lo que agregamos:
        $$S \longrightarrow aTc$$
        Uniendo las producciones anteriores para $S$ tendriamos algo asi:
        $$S \longrightarrow aSc \mid aTc$$
        Donde $T$ generaria $b^nc^n$ y $S$ generaria $a^mTc^m$, luego
        agregamos una produccion que le de fin a $T$ con:
        $$T \longrightarrow bc$$
        Finalmente escribimos la gramatica $G_1$ como:
        \begin{itemize}
            \item[] $S \longrightarrow aSc \mid aTc$
            \item[] $T \longrightarrow bTc \mid bc$
        \end{itemize}

    \subsection{Gramatica $G_2$ para $L_2$}
        Teniendo en cuenta la gramatica que genera a $\{a^nb^n \colon n \in \mathbb{N}\}$ a la cual llamaremos $G_{ab}$:
        $$S \longrightarrow aSb \mid ab$$
        Nos basta con concatenar 2 gramaticas de este tipo ya que:
        $$a^mb^mc^nd^n = a^mb^m \cdot c^nd^n$$
        $$L(a^mb^m \cdot c^nd^n) = L(G_{ab}) \cdot L(G_{cd})$$
        De esta forma nombramos a $G_{ab}$ como $A$ y a $G_{cd}$ como $B$, lo que resulta en nuestro $G_2$:
        \begin{itemize}
            \item[] $S \longrightarrow AB$
            \item[] $A \longrightarrow aAb \mid ab$
            \item[] $B \longrightarrow cBd \mid cd$
        \end{itemize}

    \subsection{Gramatica $G_3$ para $L_3$}
        Lo primero que podemos notar es que como minimo debe haber una $a$:
        $$S \longrightarrow a$$
        Luego necesitamos generar pares de simbolos alrededor para que la palabra
        mantenga un largo impar:
        $$S \longrightarrow TST$$
        Ahora juntamos las producciones anteriores en:
        $$S \longrightarrow TST \mid a$$
        Donde $T$ debiese poder ser $a$ o $b$ pero no $\varepsilon$ para poder
        generar todas las combinaciones de $a's$ y $b's$ alrededor de una $a$,
        manteniendo el largo impar. Por lo que para $T$ tenemos:
        $$T \longrightarrow a \mid b$$
        Finalmente escribimos nuestra gramatica $G_3$ como:
        \begin{itemize}
            \item[] $S \longrightarrow TST \mid a$
            \item[] $T \longrightarrow a \mid b$
        \end{itemize}

\end{document}
